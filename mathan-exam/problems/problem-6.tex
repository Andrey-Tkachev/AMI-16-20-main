\subsection{}
(этот пункт до конца не хватило времени доказать, но гуглится легко)0 )\\
Чтобы показать счетность $\mathbb{Q}$, покажем сперва счетность множества неотрицательных рациональных чисел, затем скажем, что для отрицательных рассуждение аналогичное, и что из счетности этих двух множеств вытекает, что их объединение так же счетно(пруф -- 5.в для случая с всего двумя множествами).\\
Неотрицательное рациональное число представимо в виде пары чисел -- натурального или нулевого числителя и натурального ненулевого знаменателя. Получается, есть биекция из множества положительных рациональных чисел в декартово произведение ...
\subsection{}
Чтобы показать несчетность $\mathbb{R}$, достаточно показать несчетность (0;1).\\
Предположим, интервал счетен и мы занумеровали все числа в нём каким-то образом. Тогда их можно выписать в таком виде:\\
$$x_1 = \overline{0.a_{11}a_{12}a_{13}a_{14}a_{15}...}$$
$$x_2 = \overline{0.a_{21}a_{22}a_{23}a_{24}a_{25}...}$$
$$x_3 = \overline{0.a_{31}a_{32}a_{33}a_{34}a_{35}...}$$
$$...$$
(тут $a_{ij}$ -- j-ая цифра после точки в i-ом числе)\\
Ясно, что все эти числа принадлежат указанному интервалу, кроме тех, где все цифры нули или девятки.\\
Построим теперь такое число $x$, которое тоже будет принадлежать интервалу, но будет отлично от любого из тех, что нам удалось занумеровать и выписать выше:
$$x = \overbrace{0.b_{11}b_{22}b_{33}b_{44}b_{55}...},$$
где $b_{ii} \neq a{ii}, b_{ii} \neq 0, b_{ii} \neq 9$. Тогда получается, что $x$ отличен от $x_i$ в i-й позиции после точки, т.е. отличен от любого из чисел. Значит, как бы мы не нумеровали числа этого интервала, в нем всегда можно будет найти незанумерованное.
\subsection{}
Множество, состоящее из десятичных дробей с нулевой целой частью и дробной, составленной из нулей и единичек не будет счетно по аналогии с несчетностью $\mathbb{R}$.\\
Точно так же предположим, что смогли занумеровать все числа вида, описанного в задаче, далее составим еще одно число, отличающееся от всех занумерованных в какой-то одной позиции. То есть как бы мы не нумеровали все такие числа, всегда найдется незанумерованное число.
