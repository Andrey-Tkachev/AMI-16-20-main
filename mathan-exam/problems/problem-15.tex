\begin{theorem}[Больцано-Вейерштрасс]
	Пусть последовательность $x$ ограничена, тогда можно выделить из неё сходящуюся подпоследовательность.
\end{theorem}

\begin{Proof}
    Пусть последовательность полностью содержится на отрезке $[a_0, b_0]$, такой существует в силу ограниченности, теперь разобьем его на 2 отрезка и выберем половину в которой бесконечное число элементов из $a$ (такая существует, иначе на отрезке конечное число элементов). Обозначим отрезок как $[a_1, b_1]$. Теперь для любого $n$, отрезок $[a_n, b_n]$ разобьем пополам и выберем половину на которой будет бесконечное число элементов из последовательности. Тогда длины отрезков стремятся к 0. Теперь на каждом из отрезков выберем любой элемент(не повторяющиеся, можем в силу бесконечности), тогда последовательность будет сходящейся и её предел будет равен $\lim\limits_{n \rightarrow \infty} a_n$. Что и требовалось доказать.
\end{Proof}

\begin{task}
\[x_n = \frac{n^2}{n + 1} \sin{\frac{\pi n}{2}}\]
\\
    \begin{itemize}
        \item \textbf{Последовательность не является бесконечно большой} Во всех точках вида $n = 4 k$, $x_n = 0$ противоречие.

        \item \textbf{Последовательность не ограничена}  Во всех точках вида  $n = 4 k + 1$, $x_n = \frac{(4 k + 1)^2}{4 k + 2}$ -- а она очевидно не ограничена.
    \end{itemize}
\end{task}

