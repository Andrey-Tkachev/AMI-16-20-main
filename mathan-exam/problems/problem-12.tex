Покажите, что последовательность $(1 + \frac{1}{n})^n$ имеет предел.

\begin{ther}
    Данная последовательность имеет предел (обозначаемый $e$).
\end{ther}

\begin{proof}
        Рассмотрим последовательность $x_n= (1+\frac{1}{n})^n, n \in \mathbb{N}$.

        Покажем, что последовательность ограничена и возрастает.

        Сначала докажем монотонность. Воспользуемся биномом Ньютона:

        $$(a+b)^n= a^{n}+\frac{n}{1}\cdot a^{n-1}\cdot b+\frac{n (n-1)}{1\cdot 2}\cdot a^{n-2}\cdot b^{2}+ \cdots +\frac{n (n-1) (n-2)\cdots (n- (n-1))}{1\cdot 2\cdot 3\cdot \cdots \cdot n}\cdot b^{n}.$$

        Полагая, что  $a= 1, b= \frac{1}{n}$,  получим:

        $$(1+\frac{1}{n})^{n}= 1+\frac{n}{1}\cdot \frac{1}{n}+\frac{n (n-1)}{1\cdot 2}\cdot \frac{1}{n^{2}}+$$

        $$+\frac{n (n-1) (n-2)}{1\cdot 2\cdot 3}\cdot \frac{1}{n^{3}}+ \ldots + \frac{n (n-1) (n-2)... (n- (n-1))}{1\cdot 2\cdot 3\cdot ...\cdot n}\cdot \frac{1}{n^{n}}= $$

        $$= 1+1+\frac{1}{1\cdot 2} (1-\frac{1}{n})+\frac{1}{1\cdot 2\cdot 3} (1-\frac{1}{n}) (1-\frac{2}{n})+\cdots + $$

        $$+\frac{1}{1\cdot 2\cdot 3\cdots \cdot n} (1-\frac{1}{n}) (1-\frac{2}{n})\cdots (1-\frac{n-1}{n}).$$

        $$(1+\frac{1}{n})^{n}= 1+1+\frac{1}{1\cdot 2} (1-\frac{1}{n})+\frac{1}{1\cdot 2\cdot 3} (1-\frac{1}{n}) (1-\frac{2}{n})+ \cdots + $$

        $$+ \frac{1}{1\cdot 2\cdot 3\cdots\cdot n} (1-\frac{1}{n}) (1-\frac{2}{n})\cdots (1-\frac{n-1}{n}). (*)$$

        Из равенства (*) следует, что с увеличением n  число положительных слагаемых в правой части увеличивается.

        Кроме того, при увеличении n число $\frac{1}{n}$ – убывает,
        поэтому величины $(1-\frac{1}{n}), (1-\frac{1}{n}), \cdots$ возрастают.

        Поэтому последовательность ${x_n} =  \{ (1+\frac{1}{n})^{n}\}$  — возрастающая, при этом $(1+\frac{1}{n})^{n}>2. (**)$

        Покажем, что она ограничена. Заменим каждую скобку в правой части равенства (*) на единицу. Правая часть увеличится, получим неравенство:

        $$(1+\frac{1}{n})^{n}<1+1+\frac{1}{1\cdot 2}+\frac{1}{1\cdot 2\cdot 3}+ \cdots +\frac{1}{1\cdot 2\cdot 3\cdot \cdots \cdot n}.$$

        Усилим полученное неравенство, заменив числа $3, 4, 5, \cdots, n $, стоящие в знаменателях дробей, числом 2:

        $$(1+\frac{1}{n})^{n} = 1+ (1+\frac{1}{2}+\frac{1}{2^2}+\cdots +\frac{1}{2^{n-1}}).$$

        Сумму в скобке найдем по формуле суммы членов геометрической прогрессии:

        $$1+\frac{1}{2}+\frac{1}{2^2}+ \cdots +\frac{1}{2^{n-1}} = \frac{1\cdot (1- (\frac{1}{2})^n)}{1-\frac{1}{2}}= 2 (1-\frac{1}{2^n})<2.$$

        Поэтому: $(1+\frac{1}{n})^{n}<1+2= 3. (***)$
        Итак, последовательность ограничена, при этом для $n \in \mathbb{N}$ выполняются неравенства (**) и (***):
        $$2 < (1+\frac{1}{n})^{n}<3.$$

        Тогда по теоремме Вейерштрасса о пределе монотонной ограниченной последовательности исследуемая последовательность имеет предел.
\end{proof}
