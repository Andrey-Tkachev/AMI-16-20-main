\documentclass[a4paper,12pt]{article}

\usepackage{header}

\begin{document}
	\title{Дискретная математика. Коллоквиум весна 2017. \\ Задачи}
	\author{Орлов Никита}
	\maketitle
	
	\section*{Задача 6}
	Докажите, что множество непересекающихся отрезков на прямой конечно или счетно.
	\subsection*{Решение}
	Для любого отрезка $[a_i, b_i]$ сделаем следующую операцию: возьмем достаточно больше $n \in N$ и возьмем среднее рациональных приближений сверху и снизу числа $c_i = \frac{a_i + b_i}{2}$ с точностью $\frac{1}{10^n}$. Это число лежит в интервале, так как оно больше левой границы и меньше правой границы.
	
	Так как интервалы не пересекаются, и для каждого интверала можно найти хотя бы одно рациональное число, поставим в соответствие интервалу это число. Получили биекцию из множества интервалов во множество рациональных чисел, а оно не более, чем счетно.
	
	\qed
	%подгоняем номера под свои
	\sep	
	
	
	\section*{Задача 8}
	Докажите, что биекций на множестве натуральных чисел континуум.
	
	\subsection*{Решение}
	Любую биекцию $\phi: \N \rightarrow \N$ можно записать в виде
	\[
	\begin{pmatrix}
	1 & 2 & 3 & \ldots & n & \ldots \\
	\phi(1) & \phi(2) & \phi(3) & \ldots & \phi(n) & \ldots 
	\end{pmatrix}
	\]
	
	Значит нам надо показать, что множество перестановок $\psi = \{1, 2, \ldots, n, \ldots \}$ континуально. 
	
	Определим инъекцию $\psi \mapsto 2^\N$: будем выписывать перестановку в унарном коде, где $n \mapsto \underbrace{11\ldots1}_n0$. Значит $\label{eq:1} \psi \precsim 2^\N (1)$.
	
	Определим инъекцию $2^\N \mapsto \psi$: для каждой последовательности если на $n$ месте стоит 0, то в последовательность выписываем последовательно числа $2n$ и $2n + 1$, иначе $2n + 1$ и $2n$. Следовательно, $\psi \succsim 2^\N (2)$.
	
	Получили, что из (1) и (2) следует $\psi \sim 2^\N$.
	
	
	\qed
	\sep
	\section*{Задача 24}
	Докажите, что декартово произведение перечислимых множеств перечислимо.
	\subsection*{Решение}
	
	Будем выполнять поочередно шаги вычисления перечислителей $A$ и $B$. Всякий раз, когда перечислитель выводит число, будем записывать его в соответствующий аккумулятор и выводить все пары, состоящие из этого числа и всех чисел другого аккумулятора. Мы получим все возможные пары.
	
	\sep	
	
	\section*{Задача 4}

	\subsection*{Решение}


	\sep	
	
	\section*{Задача 5}
	
	\subsection*{Решение}
	
	

	\sep	
		
		
	\section*{Задача 6}

	
	\subsection*{Решение}
	


	\sep		
	
	
	
	
	\section*{Задача 7}

	
	\subsection*{Решение}
	


	\sep	
	
	
	
	
\end{document}
