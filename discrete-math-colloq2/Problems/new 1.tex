\documentclass[a4paper,12pt]{article}

\usepackage[T1,T2A]{fontenc}        % Кодировки шрифтов
\usepackage[utf8]{inputenc}         % Кодировка текста
\usepackage[english,russian]{babel} % Подключение поддержки языков

\usepackage{amsthm}                 % Оформление теорем
\usepackage{amstext}                % Текстовые вставки в формулы
\usepackage{amsfonts}               % Математические шрифты

\newtheorem*{ther}{Теорема}
\newtheorem*{defi}{Определение}

\newcommand{\eps}{\varepsilon}
\usepackage{graphicx}

\begin{document}

    \section*{Задача 1}
    
    Вероятностное пространство: последовательности (x1, x2, x3, x4) длины 4, состоящие из целых чисел в диапазоне от 1 до 6. Все исходы равновозможны. Найдите вероятность события x1x2x3x4 четно.
    
    \subsection*{Решение}
    
    Если хотя бы одно из чисел четно, то всё произведение тоже четно. Четных исходов от 1 до 6: 2, 4, 6. Всего исходов: $6^4$. интересующих нас исходов(выбираем от одной до 4 позиций, где будут стоять четные числа, на остальные ставим нечетные) $4 * 3 * 3^3 + 6 * 3 * 3 * 3^2 + 4 * 3 * 3^3 + 3^4 = 1215. $
    
    Ответ: $1215/1296 = 0.9375$.
    
    
\end{document}