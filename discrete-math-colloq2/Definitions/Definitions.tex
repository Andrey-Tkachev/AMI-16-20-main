\documentclass[a4paper,12pt]{article}

\usepackage{header}

\begin{document}
	\title{Дискретная математика. Коллоквиум весна 2017.\\ Определения}
	\author{Потом заполню}
	\maketitle
	
	%Определения даются просто как \item.
	\begin{enumerate}
	
	%1
	 \item 1
    
    Основные понятия элементарной теории вероятностей. Исходы, события, вероятность события.

	Определение 1.
    
	Пространством элементарных исходов   ${\mathbf \Omega}$ («омега») называется множество, содержащее все возможные результаты 	данного случайного эксперимента, из которых в эксперименте происходит ровно один. Элементы этого множества называют элементарными исходами  и обозначают буквой $\omega$ («омега») с индексами или без.

	Определение 2.
	Событиями  мы будем называть подмножества множества ${\mathbf \Omega}$. Говорят, что в результате эксперимента произошло событие  	$A\subseteq \mathbf \Omega$, если в эксперименте произошел один из элементарных исходов, входящих в множество $A$.

	Определение 3.

	Поставим каждому элементарному исходу  $\omega_i\in \mathbf\Omega$ в соответствие число  $p (\omega_i)\in [0,1]$ так, что

	$\sum_{\omega_i\in \mathbf\Omega} p(\omega_i) =1.$

	Назовем число $p (\omega_i)$ вероятностью  элементарного исхода $\omega_i$. Вероятностью  события $A~\subseteq~\mathbf\Omega$ 	называется число

	${\mathsf P}(A) = \sum_{\omega_i\in A} p(\omega_i)$,
	равное сумме вероятностей элементарных исходов, входящих в множество $A$.
	
	
        \item
        %4
        События $A$ и $B$ называются независимыми, если вероятность композиции событий $p(AB)$ равна произведению вероятностей 		$p(A)\cdot p(B)$
        \medskip\\
        Свойства независимых событий: \begin{enumerate}
            \item Если $p(B)\ne0$, то условная вероятности $p(A\,|\,B)$ равна вероятности события $p(A)$.
            \item Если события $A$ и $B$ независимы, то события $\overline{A}$ и $B$, $A$ и $\overline{B}$ и $\overline{A}$ и 	$\overline{B}$ также независимы.
        \end{enumerate}
		\item 6
		Множества называются \textit{равномощными}, если между ними существует \textit{биекция}, или взаимо-однозначное 	соответствие. Равномощность множеств обозначают значком $\sim$.
		
		Свойства равномощности:
		\begin{enumerate}
			\item \textit{Cимметричтность:} $A \sim B \Rightarrow B \sim A$.
			\item \textit{Рефлексивность:} $\forall A: \ A \sim A$
			\item \textit{Транзитивность:} $A \sim B, \ B \sim C  \Rightarrow A \sim C$
		\end{enumerate}
	
		\item 8
		Множество $S$ называют \textit{счетным}, если оно равномощно множеству $\N$.	Счетные множества обладают некоторыми свойствами:
		\begin{enumerate}
			\item Объединение счетных множеств счетно
			\item Всякое подмножество счетного множества конечно или счетно
			\item Всякое бесконечное множество содержит счетное подмножество
			\item Множество $\Q$ рациональных чисел счетно
			\item 
			Конечное либо счетное объединение конечных либо счетных множеств конечно либо счетно.
			
			\item Декартово произведение счетных множеств $A \times B$ счетно.
			\item Число слов в конечном или счетном алфавите счетно.
		\end{enumerate}
		\item 9 \\
		Континуум - мощность множества [0,1]. Примеры:
		\begin{enumerate}
			\item Множество бесконечных последовательностей нулей и единиц
			\item Множество вещественных чисел
			\item Квадрат [0,1]x[0,1].
		\end{enumerate}
		\item 10
		Свойства континуума:
		\begin{enumerate}
		\item В любом континуальном множестве есть счетное подмножество.
		\item Мощность объединения не более чем континуального семейства множеств,
		каждое из которых не более чем континуально, не превосходит континуума.
		\end{enumerate}
		
		\item 11
		Булева функция от $n$ аргументов - отображение из $B^{n}$ в $B$, где B - $\{0,1\}$.
		Количество всех $n$-арных булевых функций равно $2^{2^{n}}$. Булеву функцию можно задать таблицей истинности.
		
		\item 12
		Полный базис - это такой набор, который для реализации любой сколь
		угодно сложной логической функции не потребует использования каких-либо других
		операций, не входящих в этот набор.
		Примеры полных базисов:
        
		\begin {enumerate}
		\item Конъюнкция, дизъюнкция, отрицание.
		\item Конъюнкция, отрицание.
		\item Конъюнкция, сложение по модулю два, константа один - базис Жегалкина.
		\item Штрих Шеффера (таблица истинности - 0111).
		\end {enumerate}
        
        \item 
        \textit{Булевой схемой} от $n$ переменных $x_1, \ldots, x_n$ называется последовательность булевых функций $g_1, \ldots, g_s$, в которой всякая $g_i$ или равна одной из переменных,
        или получается из предыдущих применением одной из логических операций из \textit{базиса схемы}. Также в булевой схеме задано некоторое число $m \geq 1$
        и члены последовательности $g_{s-m+1}, \ldots, g_s$ называются выходами схемы.
        Число m называют числом выходов. Число s называют размером схемы.
		\item
        %18
        Свойства вычислимой функции:
        \begin{enumerate}
            \item Если функция $f$ вычислима, то её область определения $D(f)$ является перечислимым множеством.
            \item Если функция $f$ вычислима, то её область значений $E(f)$ является перечислимым множеством.
            \item Если функция $f$ вычислима, то для любого перечислимого множества $X$ его образ $f(X)$ является перечислимым множеством.
            \item Если функция $f$ вычислима, то для любого перечислимого множества $X$ его прообраз $f^{-1}(X)$ является перечислимым множеством.
        \end{enumerate}
        \item
        %19
        Множество называется \textit{разрешимым}, если для него существует разрешающий алгоритм, который на любом входе останавливается за конечное число шагов ({\itshape разрешающий алгоритм для множества --- алгоритм, получающий на вход натуральное число и определяющий, принадлежит ли оно данному множеству}).
        \item
        %20
        Множество называется \textit{перечислимым}, если все его элементы могут быть получены с помощью некоторого алгоритма.
        \item
        %21
        Свойства перечислимых множеств:
        \begin{enumerate}
            \item Если множества $A$ и $B$ перечислимы, то их объединение $A \cup B$ и пересечение $A \cap B$ также перечислимы (\textit{отсюда следует, что объединение или пересечение конечного числа перечислимых множеств перечислимо}).
            \item Если множество $A$ перечислимо, то оно является областью значений некоторой вычислимой функции (\textit{это также является достаточным условием перечислимости}).
            \item Если множество $A$ перечислимо, то оно является областью определения некоторой вычислимой функции (\textit{это также является достаточным условием перечислимости}).
        \end{enumerate}
        \item
        %22
        Функция $U:\N\times\N\to\N$ называется универсальной, если для любой функции $f: \N\to\N$ существует такое $p$, что $U(p, x)=f(x)$ для любых $x$ (\textit{равенство здесь понимается в том смысле, что при любом $x$ обе функции либо принимают одинаковое значение, либо не определены}).
        
		\item 24
		Универсальную вычислимую функцию $U(p, x)$ называют \textit{главной}, если для любой вычислимой функции $V(q, y)$ существует \textit{транслятор} -- вычислимая тотальная функция, такая что 
		\[
		\forall q, y: \ V(q, y) = U(s(q), y)
		\]
		Такие функции также называются \textit{главными нумерациями}.
		\item 25
		Пусть $F = \{ f \ | \ f : \N \rightarrow \N \}$ - множество вычислимых функций. Пусть $A \subseteq F$ - подмножество функций. Говорят, что функция удовлетворяет некому \textit{свойству}, если она лежит в $A$. Пусть $U(p, x)$ -- универсальная функция. Пусть $P_a = \{ p \ | \ U(p, x) \in A \}$. Утверждается, что если A -- нетривиально (т.е $A \neq \oslash, \overline{A} \neq \oslash$), то множество $P_a$ неразрешимо.
    	\item 26
        Пусть $U(p, x)$ -- главная нумерация. Тогда для любой тотальной вычислимой функции $p(t) \  \exists t: \ U(p(t), x) = U(t, x)$. Это утверждение называется теоремой о неподвижной точке.
	\end{enumerate}
		
	
\end{document}
