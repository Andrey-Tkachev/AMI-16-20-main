\documentclass[a4paper,12pt]{article}

\usepackage{header}
%\setcounter{secnumdepth}{0} % sections are level 1

\begin{document}
	\title{Дискретная математика. Коллоквиум весна 2017. \\ Теоремы}
	\author{Ваномас}
	\maketitle
	\tableofcontents
    \pagebreak
    \section {Теорема 1}
    \begin{theorem}
   Пусть ${\displaystyle (\Omega ,{\mathfrak {F}},{\mathcal {P}})} (\Omega ,{\mathfrak {F}},{\mathcal {P}})$ — вероятностное пространство. Тогда для произвольных событий ${\displaystyle A_{1},A_{2},\ldots ,A_{n}} A_{1},A_{2},\ldots ,A_{n}$ справедлива формула

${\displaystyle {\mathcal {P}}{\biggl (}\bigcup _{i=1}^{n}A_{i}{\biggr )}=\sum _{i}{\mathcal {P}}(A_{i})-\sum _{i<j}{\mathcal {P}}(A_{i}\cap A_{j})+\sum _{i<j<k}{\mathcal {P}}(A_{i}\cap A_{j}\cap A_{k})+\ldots +(-1)^{n-1}\,{\mathcal {P}}\left(\bigcap _{i=1}^{n}A_{i}\right).} {\mathcal {P}}{\biggl (}\bigcup _{i=1}^{n}A_{i}{\biggr )}=\sum _{i}{\mathcal {P}}(A_{i})-\sum _{i<j}{\mathcal {P}}(A_{i}\cap A_{j})+\sum _{i<j<k}{\mathcal {P}}(A_{i}\cap A_{j}\cap A_{k})+\ldots +(-1)^{n-1}\,{\mathcal {P}}\left(\bigcap _{i=1}^{n}A_{i}\right)$.
    \end{theorem}
    \begin{proof}
    Её можно получить из принципа включений-исключений в форме индикаторных функций:

${\displaystyle \mathbf {1} _{\bigcup _{i}A_{i}}=\sum _{i}\mathbf {1} _{A_{i}}-\sum _{i<j}\mathbf {1} _{A_{i}\cap A_{j}}+\sum _{i<j<k}\mathbf {1} _{A_{i}\cap A_{j}\cap A_{k}}+\ldots +(-1)^{n-1}\,\mathbf {1} _{A_{1}\cap \ldots \cap A_{n}}.} \mathbf {1} _{\bigcup _{i}A_{i}}=\sum _{i}\mathbf {1} _{A_{i}}-\sum _{i<j}\mathbf {1} _{A_{i}\cap A_{j}}+\sum _{i<j<k}\mathbf {1} _{A_{i}\cap A_{j}\cap A_{k}}+\ldots +(-1)^{n-1}\,\mathbf {1} _{A_{1}\cap \ldots \cap A_{n}}.$

Пусть ${\displaystyle A_{i}} A_{i}$ — события вероятностного пространства ${\displaystyle (\Omega ,{\mathfrak {F}},{\mathcal {P}})} (\Omega ,{\mathfrak {F}},{\mathcal {P}})$, то есть ${\displaystyle A_{i}\in {\mathfrak {F}}} A_{i}\in {\mathfrak {F}}$. Возьмем математическое ожидание ${\displaystyle {\mathcal {M}}} {\mathcal {M}}$ от обеих частей этого соотношения, и, воспользовавших линейностью математического ожидания и равенством ${\displaystyle {\mathcal {P}}(A)={\mathcal {M}}(\mathbf {1} _{A})} {\mathcal {P}}(A)={\mathcal {M}}(\mathbf {1} _{A})$ для произвольного события ${\displaystyle A\in {\mathfrak {F}}} {\displaystyle A\in {\mathfrak {F}}}$, получим формулу включения-исключения для вероятностей.
    
    \end{proof}
    
    \sep
    \section{Теорема 2}
        \begin{theorem}
            Условную вероятность $Pr[A|B]$ можно вычислить по формуле Байеса: 
                $$ Pr[A|B] = \frac{Pr[B|A]}{Pr[B]} \cdot Pr[A]$$
        \end{theorem}

        \begin{proof}
            $$ Pr[A|B] = \frac{Pr[B|A]}{Pr[B]} \cdot Pr[A] $$
                $$\Updownarrow$$
            $$ Pr[A|B] \cdot Pr[B] = Pr[B|A] \cdot Pr[A] $$
                $$\Updownarrow$$
            $$ \frac{Pr[A\cap B]}{Pr[B]} \cdot Pr[B] = \frac{Pr[B\cap A}{Pr[A]} \cdot Pr[A] $$
                $$\Updownarrow$$
            $$Pr[A\cap B] = Pr[B\cap A]$$

            Т.к. $A\cap B = B\cap A$, то последнее равенство верно, а значит верна формула Байеса. 
        \end{proof}

    \sep
    \section {Теорема 3}
    \begin{theorem}
   Условной вероятностью события A при условии события B называется

${\displaystyle \mathbb {P} (A\mid B)={\frac {\mathbb {P} (A\cap B)}{\mathbb {P} (B)}}} \mathbb{P}(A \mid B) = \frac{\mathbb{P}(A\cap B)}{\mathbb{P}(B)}$, где ${\displaystyle \mathbb {P} (A\cap B)} {\displaystyle \mathbb {P} (A\cap B)}$ — вероятность наступления обоих событий сразу.
    \end{theorem}
    \begin{proof}
    Пусть ровно r исходов события B входят и в событие А. Исходы события В уже реализовались
    В данном испытании произошло одно из t событий, входящих в B. Все элементарные события равновероятны, следовательно, для данного испытания вероятность наступления произвольного элементарного события, входящего в B равна 1/t. Тогда по классическому определению вероятности, в данном испытании событие A произойдет с вероятностью r/t.
    $P(A|B) = \frac{\frac{r}{m}}{\frac{t}{m}} = \frac{P(AB)}{P{B}}$

    
    \end{proof}
    \sep
	\section{Теорема 4}
	\begin{theorem}
		Математическое ожидание $E$ линейно.
	\end{theorem}
	\begin{proof}
	Пусть $\xi$ и $\eta$ --- случайные величины, заданные на одном вероятностном пространстве. Тогда выполняется равенство $$E(\xi+\eta)=\sum_w(\xi(w)+\eta(w))p(w)=\sum_w\xi(w)p(w)+\sum_w\eta(w)p(w)=E(\xi)+E(\eta)$$
	То есть математическое ожидание суммы случайных величин равно сумме математического ожидания каждой из этих величин. Пусть теперь $\xi$ --- случайная величина, $\alpha$ --- действительное число. Тогда выполняется равенство 
	$$E(\alpha\cdot\xi)=\sum_w(\alpha\cdot\xi(w)p(w))=\alpha\cdot\sum_w\xi(w)p(w)=\alpha\cdot E(\xi)$$
	То есть математическое ожидание произведения константы и случайной величины равно произведению этой константы и математического ожидания самой величины.\\
	Таким образом, линейность математического ожидания доказана.
	\end{proof}
\sep
    \section {Теорема 5}
    \begin{theorem}
   Неравенство Маркова в теории вероятностей дает оценку вероятности, что случайная величина превзойдет по модулю фиксированную положительную константу, в терминах её математического ожидания. Получаемая оценка обычно груба, однако она позволяет получить определённое представление о распределении, когда последнее не известно явным образом.
   
   Пусть случайная величина X: $\Omega \rightarrow \mathbb R\mathrm+$ определена на вероятностном пространстве $(\Omega, F, \mathbb R)$, и ее математическое ожидание $\mathbb E\mathrm |\xi|<\mathcal {1}$. Тогда
 $\forall ~x > 0~~ \mathbb P\mathrm(|\xi| \ge x)\le \frac {\mathbb E\mathrm |\xi|}{x}$
 
    \end{theorem}
    \begin{proof}
    
    Возьмем для доказательства следующее понятие:
    
Пусть A - некоторое событие. Назовем индикатором события A случайную величину I, равную единице если событие A произошло, и нулю в противном случае. По определению величина I(A) имеет распределение Бернулли с параметром

 $p = \mathbb P\mathrm (I(A) = 1) = \mathbb P\mathrm (A)$, 
 
и ее математическое ожидание равно вероятности успеха  $p = \mathbb P\mathrm (A)$. Индикаторы прямого и противоположного событий связаны равенством $I(A) + I(\overline A) = 1$. Поэтому
 $|\xi|=|\xi|*I(|\xi|<x)+|\xi|*I(|\xi|\ge x)\ge |\xi|*I(|\xi|\ge x)\ge x*I(|\xi| \ge x)$.
Тогда
 $\mathbb E\mathrm |\xi|\ge \mathbb E\mathrm(x*I(|\xi|\ge x)) = x*\mathbb P\mathrm (|\xi|\ge x)$.
 
Разделим обе части на x:

$\mathbb P\mathrm(|\xi| \ge x)\le \frac {\mathbb E\mathrm |\xi|}{x}$

Пример:

Ученики в среднем опаздывают на 3 минуты. Какова вероятность того, что ученик опоздает на 15 минут и более? Дать грубую оценку сверху.

 $\mathbb P\mathrm (|\xi|\ge 15)\le 3/15 = 0.2$
    
    \end{proof}
    \sep
    \section {Теорема 7}
    \begin{theorem}
   
Объединение счетного числа счетных или конечных множеств счетно или конечно
    \end{theorem}
    \begin{proof}
    
    Пусть имеется счётное число счётных множеств A1, A2, . . .
    
Расположив элементы каждого из них слева направо в последовательность (Ai = {ai0, ai1, . . . }) и поместив эти последовательности друг под другом, получим таблицу

a00 a01 a02 a03 . . .

a10 a11 a12 a13 . . .

a20 a21 a22 a23 . . .

a30 a31 a32 a33 . . .

. . . . . . . . . . . . . . .

Теперь эту таблицу можно развернуть в последовательность, например, проходя по очереди диагонали:
a00, a01, a10, a02, a11, a20, a03, a12, a21, a30, . . .
Если множества Ai не пересекались, то мы получили искомое представление для их объединения. 

Если пересекались, то из построенной последовательности надо выбросить повторения.
Если множеств конечное число или какие-то из множеств конечны, то в этой конструкции части членов не будет — и останется либо конечное, либо счётное множество.
    
    \end{proof}

    \sep
    \section{Теорема 14}
        \begin{theorem}
            Верхняя оценка $O(n2^n)$ схемной сложности булевой функции от $n$ переменных.
        \end{theorem}

        \begin{proof}
            Для всякого $a \in {0, 1}^n$ рассмотрим функцию $f_a : {0, 1}^n \rightarrow {0, 1}$,
            такую что $f_a(x) = 1$ тогда и только тогда, когда $x = a$. Будет удобно ввести обозначение $x^1 = x$ и $x_0 = \neg x$. Тогда функцию $f_a$ можно записать формулой
                
                    $$ f_a(x) = \bigwedge_{i = 1}^{n} x^{a_i}_i,$$
            
            где $x = (x_1, \ldots , x_n)$ и $a = (a_1, \ldots , a_n)$.
            Для произвольной функции $f$ уже не сложно записать формулу через функции
                
                $$ f(x) = \bigvee_{a \in f^{-1}(1)} f_a(x).$$

            Теперь эти формулы можно переделать в схему. Наша схема сначала будет вычислять отрицания всех переменных, на это нужно n элементов. После этого можно
            вычислить все функции $f_a$. Для вычисления каждого нужно $n - 1$ раз применить
            конъюнкцию. Всего получается $2^n(n -1)$ элемент. Наконец, для вычисления $f$ нужно взять дизъюнкцию нужных функций $f_a$, на это уйдет не более $2^n$ элементов (всего различных функций $f_a$ ровно $2^n$ -- над каждым аргументом отрицание либо есть, либо нет).
            Суммарно в нашей схеме получается $O(n2^n)$ элементов. 
        \end{proof}

    
    \sep
    
    \section {Теорема 18}
    \begin{theorem}
    Множество $M$ и его дополнение $\overline{M}$ разрешимы тогда и только тогда, когда $M$ и $\overline{M}$ перечислимы.
    \end{theorem}
    \begin{proof}
    \strut\\\\\textit{Необходимость:}\smallskip\\
    Пусть $M$ и $\overline{M}$ разрешимы. Случаи, когда $M=\N$ или $M=\varnothing$, тривиальны. Будем считать, что $M\ne\varnothing$ и $\M\ne\N$. Тогда существуют такие $a$ и $b$, что $a\in M$ и $b\in\overline{M}$. Поскольку $M$ разрешимо, его характеристическая функция $\chi_M$ вычислима. Рассмотрим функцию 
    $$f(x) = 
    \begin{cases}
    x \text{ при } \chi_M(x) = 1 \\
    a \text{ при } \chi_M(x) = 0 \\
    \end{cases}
    $$
    $M$ является множеством значений $f$: ничего, кроме значений $M$, в $E(f)$, очевидно, быть не может, а для любого $m\in M$ верно, что $f(m)=m$. Аналогично, рассмотрим функцию
    $$g(x) = 
    \begin{cases}
    x \text{ при } \chi_M(x) = 0 \\
    b \text{ при } \chi_M(x) = 1 \\
    \end{cases}
    $$
    $\overline{M}$ является областью значений $g$. Таким образом, $M$ и $\overline{M}$ перечислимы (перечисляющие алгоритмы могут быть, например, устроены так: последовательно для всех натуральных $n$, начиная с нуля, алгоритм выводит значение $f(n)$ или $g(n)$ соответственно).\\\\
    \textit{Достаточность:}\smallskip\\
    Пусть $M$ и $\overline{M}$ перечислимы. Тогда существуют алгоритмы соответственно $\mathfrak{A}$ и $\mathfrak{B}$, с помощью которых могут быть получены все элементы этих множеств. Рассмотрим алгоритм, запускающий $\mathfrak{A}$ и $\mathfrak{B}$ паралленьно, который выводит сначала первое число, полученное $\mathfrak{A}$, затем --- первое число, полученное $\mathfrak{B}$, затем --- второе число, полученное $\mathfrak{A}$, и так далее. Такой алгоритм будет являться перечисляющим алгоритмом $\N$, который получает элементы $M$ на нечётных выводах и элементы $\overline{M}$ --- на чётных. Соответсвенно, для любого элемента $x$ верно, что он будет выведен рассматриваемым алгоритмом за конечное число шагов. Если он был выведен как нечётный по счёту вывод, то $\chi_M(x)=1$, если как чётный --- $\chi_M(x)=0$. Таким образом, $\chi_M$ вычислима, а значит, $M$ и $\overline{M}$ разрешимы.
    \end{proof}
    
    \sep
    
    \section {Теорема 19}
    \begin{theorem}
    Перечислимые множества являются множествами значений вычислимых функций.
    \end{theorem}
    \begin{proof} Пусть $M$ --- перечислимое множество. Тогда существует алгоритм $\mathfrak{A}$, выводящий все его элементы. Рассмотрим алгоритм, который принимает на вход натуральное число $n$, после чего запускает $\mathfrak{A}$ и считает его выводы. Дойдя до $n$-го по счёту (начиная с 0) вывода, алгоритм останавливается, выводя $n$-й вывод алгоритма $\mathfrak{A}$ как результат своей работы.\\
    Множество значений функции, которую вычисляет вышеописанный алгоритм, будет совпадать с множеством чисел, выводимых $\mathfrak{A}$, то есть с $M$.
    
    \end{proof}
    
    \sep
    
    \section {Теорема 20}
    \begin{theorem}
    Перечислимые множества являются множествами значений всюду определённых вычислимых функций.
    \end{theorem}
    \begin{proof} Пусть $M$ --- перечислимое множество. Тогда существует алгоритм $\mathfrak{A}$, выводящий все его элементы. Рассмотрим алгоритм, который принимает на вход натуральное число $n$, после чего запускает $\mathfrak{A}$ и считает его выводы. Дойдя до $n$-го по счёту (начиная с 0) вывода, алгоритм останавливается, выводя $n$-й вывод алгоритма $\mathfrak{A}$ как результат своей работы.\\
    Множество значений функции $f$, которую вычисляет вышеописанный алгоритм, будет совпадать с множеством чисел, выводимых $\mathfrak{A}$, то есть с $M$. Если множество $M$ бесконечно, то $f$ также будет всюду определённой по построению. Если же $M$ конечно, рассмотрим функцию $f_1(x)=f(x \text{ mod } (l + 1))$, где $l$ --- номер вывода $\mathfrak{A}$, после которого количество различных выведенных $\mathfrak{A}$ элементов станет равно $[M]$. Значение $l$ будет конечным, так как любой элемент $M$ выводится $\mathfrak{A}$ за конечное число шагов. Данная функция будет всюду определённой, поскольку $\mathfrak{A}$ до своей остановки совершает не менее $l$ шагов, и множество её значений будет совпадать с $M$, поскольку по построению в множестве её значений $[M]$ различных элементов, и все они являются результатом работы $\mathfrak{A}$.
    
    \end{proof}
    
    \sep
    
    \section {Теорема 21}
    \begin{theorem}
    Множества значений всюду определённых функций перечислимы.
    \end{theorem}
    \begin{proof} 
    Пусть $M=f(\N)$ --- множество значений некоторой всюду определённой функции $f$. Рассмотрим алгоритм, последовательно выводящий для каждого натурального числа $n$, начиная с $0$, значение $f(n)$. Он будет являться перечисляющий алгоритмом для $M$: для любого $m\in M$ верно, что $\exists x \in \N : f(x) = m$, следовательно, вышеописанный алгоритм выведет $m$ на своём $x$-ом шаге.
    \end{proof}
	
	\sep
    
    \section {Теорема 22}
    \begin{theorem}
    Множество значений всюду опрелённой вычислимой функции является областью определения вычислимой функции.
    \end{theorem} 
    \begin{proof} Пусть $f$ --- всюду определённая вычислимая функция. Рассмотрим алгоритм, принимающий на вход натуральное число $x$, который последовательно вычисляет значения $f(n)$ для всех натуральных $n$, начиная с $0$, и, если полученное в какой-то момент значение равно $x$, выводит $1$. Если $x\in E(f)$, то $\exists m \in \N : f(m) = x$. Тогда вышеописанный алгоритм остановится за конечное число шагов: он завершит свою работу, вычислив значения $f(n)$ для всех $n\leqslant m$, а для этого требуется конечное число шагов, поскольку $f$ вычислима и всюду определена. Если же $x\not\in E(f)$, то данный алгоритм никогда не остановится, поскольку условие его остановки --- существование такого $m\in\N$, что $f(m)=x$. Таким образом, функция, вычисляемая вышеопианным алгоритмом, определена в точности на $E(f)$.
    \end{proof}
	
	\sep
	
\end{document}
