\documentclass[a4paper,12pt]{article}

\usepackage{header}
%\setcounter{secnumdepth}{0} % sections are level 1

\begin{document}
	\title{Дискретная математика. Коллоквиум весна 2017. \\ Теоремы}
	\author{Ваномас}
	\maketitle
	\tableofcontents
    \pagebreak
	\section{Теорема 4}
	\begin{theorem}
		Математическое ожидание $E$ линейно.
	\end{theorem}
	\begin{proof}
	Пусть $\xi$ и $\eta$ --- случайные величины, заданные на одном вероятностном пространстве. Тогда выполняется равенство $$E(\xi+\eta)=\sum_w(\xi(w)+\eta(w))p(w)=\sum_w\xi(w)p(w)+\sum_w\eta(w)p(w)=E(\xi)+E(\eta)$$
	То есть математическое ожидание суммы случайных величин равно сумме математического ожидания каждой из этих величин. Пусть теперь $\xi$ --- случайная величина, $\alpha$ --- действительное число. Тогда выполняется равенство 
	$$E(\alpha\cdot\xi)=\sum_w(\alpha\cdot\xi(w)p(w))=\alpha\cdot\sum_w\xi(w)p(w)=\alpha\cdot E(\xi)$$
	То есть математическое ожидание произведения константы и случайной величины равно произведению этой константы и математического ожидания самой величины.\\
	Таким образом, линейность математического ожидания доказана.
	\end{proof}
    
    \sep
    
    \section {Теорема 18}
    \begin{theorem}
    Множество $M$ и его дополнение $\overline{M}$ разрешимы тогда и только тогда, когда $M$ и $\overline{M}$ перечислимы.
    \end{theorem}
    \begin{proof}
    \strut\\\\\textit{Необходимость:}\smallskip\\
    Пусть $M$ и $\overline{M}$ разрешимы. Случаи, когда $M=\N$ или $M=\varnothing$, тривиальны. Будем считать, что $M\ne\varnothing$ и $\M\ne\N$. Тогда существуют такие $a$ и $b$, что $a\in M$ и $b\in\overline{M}$. Поскольку $M$ разрешимо, его характеристическая функция $\chi_M$ вычислима. Рассмотрим функцию 
    $$f(x) = 
    \begin{cases}
    x \text{ при } \chi_M(x) = 1 \\
    a \text{ при } \chi_M(x) = 0 \\
    \end{cases}
    $$
    $M$ является множеством значений $f$: ничего, кроме значений $M$, в $E(f)$, очевидно, быть не может, а для любого $m\in M$ верно, что $f(m)=m$. Аналогично, рассмотрим функцию
    $$g(x) = 
    \begin{cases}
    x \text{ при } \chi_M(x) = 0 \\
    b \text{ при } \chi_M(x) = 1 \\
    \end{cases}
    $$
    $\overline{M}$ является областью значений $g$. Таким образом, $M$ и $\overline{M}$ перечислимы (перечисляющие алгоритмы могут быть, например, устроены так: последовательно для всех натуральных $n$, начиная с нуля, алгоритм выводит значение $f(n)$ или $g(n)$ соответственно).\\\\
    \textit{Достаточность:}\smallskip\\
    Пусть $M$ и $\overline{M}$ перечислимы. Тогда существуют алгоритмы соответственно $\mathfrak{A}$ и $\mathfrak{B}$, с помощью которых могут быть получены все элементы этих множеств. Рассмотрим алгоритм, запускающий $\mathfrak{A}$ и $\mathfrak{B}$ паралленьно, который выводит сначала первое число, полученное $\mathfrak{A}$, затем --- первое число, полученное $\mathfrak{B}$, затем --- второе число, полученное $\mathfrak{A}$, и так далее. Такой алгоритм будет являться перечисляющим алгоритмом $\N$, который получает элементы $M$ на нечётных выводах и элементы $\overline{M}$ --- на чётных. Соответсвенно, для любого элемента $x$ верно, что он будет выведен рассматриваемым алгоритмом за конечное число шагов. Если он был выведен как нечётный по счёту вывод, то $\chi_M(x)=1$, если как чётный --- $\chi_M(x)=0$. Таким образом, $\chi_M$ вычислима, а значит, $M$ и $\overline{M}$ разрешимы.
    \end{proof}
    
    \sep
    
    \section {Теорема 19}
    \begin{theorem}
    Перечислимые множества являются множествами значений вычислимых функций.
    \end{theorem}
    \begin{proof} Пусть $M$ --- перечислимое множество. Тогда существует алгоритм $\mathfrak{A}$, выводящий все его элементы. Рассмотрим алгоритм, который принимает на вход натуральное число $n$, после чего запускает $\mathfrak{A}$ и считает его выводы. Дойдя до $n$-го по счёту (начиная с 0) вывода, алгоритм останавливается, выводя $n$-й вывод алгоритма $\mathfrak{A}$ как результат своей работы.\\
    Множество значений функции, которую вычисляет вышеописанный алгоритм, будет совпадать с множеством чисел, выводимых $\mathfrak{A}$, то есть с $M$.
    
    \end{proof}
    
    \sep
    
    \section {Теорема 20}
    \begin{theorem}
    Перечислимые множества являются множествами значений всюду определённых вычислимых функций.
    \end{theorem}
    \begin{proof} Пусть $M$ --- перечислимое множество. Тогда существует алгоритм $\mathfrak{A}$, выводящий все его элементы. Рассмотрим алгоритм, который принимает на вход натуральное число $n$, после чего запускает $\mathfrak{A}$ и считает его выводы. Дойдя до $n$-го по счёту (начиная с 0) вывода, алгоритм останавливается, выводя $n$-й вывод алгоритма $\mathfrak{A}$ как результат своей работы.\\
    Множество значений функции $f$, которую вычисляет вышеописанный алгоритм, будет совпадать с множеством чисел, выводимых $\mathfrak{A}$, то есть с $M$. Если множество $M$ бесконечно, то $f$ также будет всюду определённой по построению. Если же $M$ конечно, рассмотрим функцию $f_1(x)=f(x \text{ mod } (l + 1))$, где $l$ --- номер вывода $\mathfrak{A}$, после которого количество различных выведенных $\mathfrak{A}$ элементов станет равно $[M]$. Значение $l$ будет конечным, так как любой элемент $M$ выводится $\mathfrak{A}$ за конечное число шагов. Данная функция будет всюду определённой, поскольку $\mathfrak{A}$ до своей остановки совершает не менее $l$ шагов, и множество её значений будет совпадать с $M$, поскольку по построению в множестве её значений $[M]$ различных элементов, и все они являются результатом работы $\mathfrak{A}$.
    
    \end{proof}
    
    \sep
    
    \section {Теорема 21}
    \begin{theorem}
    Множества значений всюду определённых функций перечислимы.
    \end{theorem}
    \begin{proof} 
    Пусть $M=f(\N)$ --- множество значений некоторой всюду определённой функции $f$. Рассмотрим алгоритм, последовательно выводящий для каждого натурального числа $n$, начиная с $0$, значение $f(n)$. Он будет являться перечисляющий алгоритмом для $M$: для любого $m\in M$ верно, что $\exists x \in \N : f(x) = m$, следовательно, вышеописанный алгоритм выведет $m$ на своём $x$-ом шаге.
    \end{proof}
	
	\sep
    
    \section {Теорема 22}
    \begin{theorem}
    Множество значений всюду опрелённой вычислимой функции является областью определения вычислимой функции.
    \end{theorem} 
    \begin{proof} Пусть $f$ --- всюду определённая вычислимая функция. Рассмотрим алгоритм, принимающий на вход натуральное число $x$, который последовательно вычисляет значения $f(n)$ для всех натуральных $n$, начиная с $0$, и, если полученное в какой-то момент значение равно $x$, выводит $1$. Если $x\in E(f)$, то $\exists m \in \N : f(m) = x$. Тогда вышеописанный алгоритм остановится за конечное число шагов: он завершит свою работу, вычислив значения $f(n)$ для всех $n\leqslant m$, а для этого требуется конечное число шагов, поскольку $f$ вычислима и всюду определена. Если же $x\not\in E(f)$, то данный алгоритм никогда не остановится, поскольку условие его остановки --- существование такого $m\in\N$, что $f(m)=x$. Таким образом, функция, вычисляемая вышеопианным алгоритмом, определена в точности на $E(f)$.
    \end{proof}
	
	\sep
	
\end{document}
