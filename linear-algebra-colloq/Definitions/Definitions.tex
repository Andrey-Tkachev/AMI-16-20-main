\documentclass[a4paper,12pt]{article}

%% Начало шапки

%% Настройка поддержки русского языка
\usepackage{cmap}                   % Поиск по кириллице
\usepackage{mathtext}               % Кириллица в формулах
\usepackage[T1,T2A]{fontenc}        % Кодировки шрифтов
\usepackage[utf8]{inputenc}         % Кодировка текста
\usepackage[english,russian]{babel} % Подключение поддержки языков

%% Настройка размеров полей
\usepackage[top=0.7in, bottom=0.75in, left=0.625in, right=0.625in]{geometry}

%% Математические пакеты
\usepackage{mathtools}              % Тот же amsmath, только с некоторыми поправками
\usepackage{amssymb}                % Математические символы
\usepackage{amsthm}                 % Оформление теорем
\usepackage{amstext}                % Текстовые вставки в формулы
\usepackage{amsfonts}               % Математические шрифты
\usepackage{icomma}                 % "Умная" запятая: $0,2$ --- число, $0, 2$ --- перечисление
\usepackage{enumitem}               % Для выравнивания itemize (\begin{itemize}[align=left])
\usepackage{array}                  % Таблицы и матрицы
\usepackage{multirow}

%% Алгоритмические пакеты и их настройки
\usepackage{algorithm}              % Шапка алгоритма
\usepackage{algorithmicx}           % Написание алгоритмов
\usepackage[noend]{algpseudocode}   % Написание псевдокода; убраны end
\usepackage{listings}               % Для кода на каком-либо языке программиования
\renewcommand{\algorithmicrequire}{\textbf{Ввод:}}              % Ввод
\renewcommand{\algorithmicensure}{\textbf{Вывод:}}              % Вывод
\floatname{algorithm}{Алгоритм}                                 % Название алгоритма
\renewcommand{\algorithmiccomment}[1]{\hspace*{\fill}\{// #1\}} % Комментарии
\newcommand{\algname}[1]{\textsc{#1}}                           % Вызов функции в алгоритме

\newcommand*{\hm}[1]{#1\nobreak\discretionary{}
	{\hbox{$\mathsurround=0pt #1$}}{}}

%% Шрифты
\usepackage{euscript}               % Шрифт Евклид
\usepackage{mathrsfs}               % \mathscr{}

%% Графика
\usepackage[pdftex]{graphicx}       % Вставка картинок
\graphicspath{{images/}}            % Стандартный путь к картинкам
\usepackage{tikz}                   % Рисование всего
\usepackage{pgfplots}               % Графики
\usetikzlibrary{calc,matrix}

%% Прочие пакеты
\usepackage{indentfirst}                    % Красная строка в начале текста
\usepackage{epigraph}                       % Эпиграфы
\usepackage{fancybox,fancyhdr}              % Колонтитулы
\usepackage[colorlinks=true, urlcolor=blue]{hyperref}   % Ссылки
\usepackage{titlesec}                       % Изменение формата заголовков
\usepackage[normalem]{ulem}                 % Для зачёркиваний
\usepackage[makeroom]{cancel}               % И снова зачёркивание (на этот раз косое)

%% Прочее
\mathtoolsset{showonlyrefs=true}        % Показывать номера только у тех формул,
% на которые есть \eqref{} в тексте.
\renewcommand{\headrulewidth}{1.8pt}    % Изменяем размер верхнего отступа колонтитула
\renewcommand{\footrulewidth}{0.0pt}    % Изменяем размер нижнего отступа колонтитула

%Прочее
\usepackage{forest} % Деревья

\renewcommand{\Re}{\mathrm{Re\:}}
\renewcommand{\Im}{\mathrm{Im\:}}
\newcommand{\Arg}{\mathrm{Arg\:}}
\renewcommand{\arg}{\mathrm{arg\:}}
\newcommand{\Mat}{\mathrm{Mat}}
\newcommand{\M}{\mathrm{M}}
\newcommand{\id}{\mathrm{id}}
\newcommand{\isom}{\xrightarrow{\sim}} 
\newcommand{\leftisom}{\xleftarrow{\sim}}
\newcommand{\Hom}{\mathrm{Hom}}
\newcommand{\Ker}{\mathrm{Ker}\:}
\newcommand{\rk}{\mathrm{rk}\:}
\newcommand{\diag}{\mathrm{diag}}
\newcommand{\ort}{\mathrm{ort}}
\newcommand{\pr}{\mathrm{pr}}
\newcommand{\vol}{\mathrm{vol\:}}
\newcommand{\Tr}{\mathrm{tr\:}}
\newcommand{\sgn}{\mathrm{sgn\:}}
\newcommand{\al}{\alpha}

%% Определения
\newtheorem{definition}{Определение}
\newtheorem*{defin}{Определение}
\newtheorem{Def}{Определение}
\newtheorem*{Lemma}{Лемма}
\newtheorem{Suggestion}{Предложение}
\newtheorem*{Examples}{Пример}
\newtheorem*{Consequence}{Следствие}
\newtheorem{Theorem}{Теорема}
\newtheorem{Statement}{Утверждение}
\newtheorem*{Task}{Упражнение}
\newtheorem*{Designation}{Обозначение}
\newtheorem*{Generalization}{Обобщение}
\newtheorem*{Thedream}{Предел мечтаний}
\newtheorem*{Properties}{Свойства}
\newtheorem*{Note}{Замечание}

\newcommand{\Z}{\mathbb{Z}}
\newcommand{\N}{\mathbb{N}}
\newcommand{\Q}{\mathbb{Q}}
\newcommand{\R}{\mathbb{R}}
\renewcommand{\C}{\mathbb{C}}
\renewcommand{\L}{\mathscr{L}}
\renewcommand{\epsilon}{\varepsilon}
\renewcommand{\phi}{\varphi}
\newcommand{\e}{\mathbb{e}}
\renewcommand{\l}{\lambda}
\newcommand{\E}{\mathbb{E}}
\def\eps{\varepsilon}
\def\limref#1#2{{#1}\negmedspace\mid_{#2}}
\newcommand{\vvector}[1]{\begin{pmatrix}{#1}_1 \\\vdots\\{#1}_n\end{pmatrix}}
\renewcommand{\vector}[1]{({#1}_1, \ldots, {#1}_n)}

\definecolor{Gray}{gray}{0.9}
\newcolumntype{g}{>{\columncolor{Gray}}c}

\newtheorem*{canonther}{Теорема о приведении матрицы к каноническому виду}
\newtheorem*{lem}{Лемма}
\newtheorem*{lem1}{Лемма 1}
\newtheorem*{lem2}{Лемма 2}
\newtheorem*{lem3}{Лемма 3}



\begin{document}
	\title{Авель. Определения. Сделано по Каину.}
	\author{ПМИ 2016 \\ Орлов Никита}
	\maketitle
	 
	Выражается отдельная благодарность Вадиму Гринбергу за предоставленный материал, а так же всем нашим однокурсникам за помощь в разработке этого сборника
	
	\begin{enumerate}
		\item СЛУ называется совместной, если у неё есть решения (хотя бы одно), и несовместной в противном случае.
		\item Две СЛУ эквивалентны тогда и только тогда, когда имеют одинаковое множество решений.
		\item Расширенная матрица системы линейный уравнений это матрица, вида $(A|B)$, где $A$ - матрица коэффициентов, $B$ - столбец свободных членов.
		
		Пример:
		\[
		\left(
			\begin{array}{ccc|c}
			a_{11} & \ldots & a_{1m} & b_1 \\
			\vdots & \ddots & \vdots & \vdots \\
			a_{n1} & \ldots & a_{nm} & b_n
			\end{array}
		\right)
		\]
		
		\item Элементарными преобразованиями строк матрицы называют:
		\subitem 1) прибавление к любой строке матрицы другой, умноженной на ненулевое число.
		\subitem 2) перестановку местами любых двух строк матрицы;
		\subitem 3) умножение любой строки матрицы на константу $k, k \neq 0$ (при этом определитель
		матрицы увеличивается в $k$ раз);
		
		\item Матрица называется ступенчатой, или имеющей ступенчатый вид, если:
		\subitem 1) номера ведущих элементов строк строго возрастают;
		\subitem 2) все нулевые строки стоят в конце (внизу).
		
		\item Матрица называется улучшенной ступенчатой, или канонической, если:
		\subitem 1) она имеет ступенчатый вид;
		\subitem 2) все ведущие элементы равны 1 и это единственный ненулевой элемент в столбцах, где они стоят
		
		\item Теорема о приведении матрицы к каноническому виду. Всякую матрицу элементарными преобразованиями можно привести к канонической.

		\item Общее решение СЛУ - запись решения, в котором базисные переменные выражены через свободные. 
		
		\item У СЛУ с действительными коэффициентами может быть 0, 1 или бесконечно много решений.
		
		\item Однородная СЛУ - система, где столбец свободных членов - нулевой:
		\[
		A\overrightarrow{x} = \overrightarrow{0}
		\Leftrightarrow
		\left(
		\begin{array}{ccc|c}
		a_{11} & \ldots & a_{1m} & 0 \\
		\vdots & \ddots & \vdots & \vdots \\
		a_{n1} & \ldots & a_{nm} & 0
		\end{array}
		\right)
		\]
		
		У такой системы может быть либо одно решение, либо бесконечно много. 
		
		\item Однородная система линейных уравнений, в которой число уравнений меньше числа неизвестных, имеет ненулевое решение. В общем же случае это верно не всегда.
		
		\item Пусть есть матрицы $A, B \in Mat_{n \times m}$. Тогда определена операция сложения матриц $A + B$:
		\[
		C = A + B
		\]
		\[
		c_{ij} = a_{ij} + b_{ij}
		\]
		
		Пусть есть число $\alpha \in \mathbb{R}$. Тогда определена операция умножения матрицы на скаляр $\alpha \cdot A$:
		\[
		D = \alpha \cdot A
		\]
		\[
		d_{ij} = \alpha \cdot a_{ij}
		\]
		
		\item Транспонированием матрицы $A \in Mat_{n \times m}$ называется такая матрица $A^T \in Mat_{m \times n}$, в которой строки записаны в виде столбцов:
		\[
		a_{ij} = a^T_{ji}
		\]
		
		\item Пусть есть матрицы $A \in Mat_{n \times k}, B \in Mat_{k \times m}$. Произведением матриц $A \times B$ будет называться матрица $C \in Mat_{n \times m }$, с элементами 
		\[
		c_{ij} = \sum_{r = 1}^{k}a_{ir}b_{rj}
		\]
		
		\item Диагональной матрицей будет квадратная матрица, где вне главной диагонали стоят нули.
		\[
		diag(a_{11}, \ldots, a_{nn}) = 
		\begin{pmatrix}
		a_{11} & 0 & \ldots & 0\\
		0 & a_{22} & \ldots & 0 \\
		\vdots &  & \ddots & \vdots \\
		0 & 0 & 0 &a_{nn}
		\end{pmatrix}
		\]
		
		\item Матрица $A \in M_n$ называется единичной, если она диагональная и на диагонали стоят единицы.
		
		\[
		diag(1, \ldots, 1) = 
		\begin{pmatrix}
			1 & 0 & \cdots & 0 \\
			0 & 1 & \cdots & 0 \\
			\vdots && \ddots & \vdots \\
			0 & 0 & \cdots & 1
		\end{pmatrix}
		\]
		Умножение единичной матрицы слева и справа на произвольную квадратную матрицу $A$ той же размерности, она не меняется:
		\[
		EA = AE = A
		\]
		
		\item Следом квадратной матрицы $A$ является сумма всех диагональных элементов матрицы $A$.
		\[
		trA = \sum_{i = 1}^{n}a_{ii}
		\]
		
		\item Пусть есть матрицы $A \in Mat_{n \times m}, B \in Mat_{m \times n}$. Тогда
		
		\[
		trAB = trBA
		\]
		
		\item Перестановкой из $n$ элементов на множестве $\{1, 2, \ldots, n\}$ является любой упорядоченный набор $n$ элементов, где каждый встречается ровно один раз. Число таких перестановок - $n!$.
		
		Подстановкой является биективное отображение множества $\{1, 2, \ldots , n\}$ в себя.
		
		\item Пусть есть подстановка $\sigma$. Тогда инверсией будет называться такая пара индексов $i, j$, что 
		\[
		i < j \Rightarrow \sigma(i) > \sigma(j)
		\]
		
		Пусть $k$ - число таких инверсий в $\sigma$. Тогда знаком посдтановки будет:
		\[
		sgn(\sigma) = -1^k
		\]
		\item Пусть есть две подстановки $\sigma$ и $\tau$. Тогда умножением подстановок будет их композиция:
		
		\[
		\sigma \circ \tau = \sigma(\tau(n))
		\]
		
		\item Тождественная подстановка - такая подстановка $id$, которая переводит элементы сами в себя:
		\[
		\forall i: id(i) = i
		\]
		
		Обратная подстановка - такая подстановка к данной $\sigma$, что \[
		\sigma \circ \sigma^{-1} = id
		\]
		
		\item Транспозиция - такая подстановка, что меняет два элемента местами.
		
		Элементарная транспозиция - такая транспозиция, что меняет два \textit{соседних} элемента местами.
		
		\item При умножении транспозиции на подстановку, у подстановки меняется знак на противоположный:
		\[
			sgn(\sigma \tau) = -sgn(\sigma)
		\]
		
		\item Знак произведения подстановок есть произведение знаков подстановок
		\[
			sgn(\sigma \rho) = sgn(\sigma) \cdot sgn(\rho)
		\]
		
		\item Пусть есть матрица $A \in M_n$. Определителем матрицы назовем число, которое считается по формуле:
		\[
		detA = \sum_{\sigma \in S_n} sgn(\sigma) \prod_{i = 1}^n\cdot a_{i\sigma(i)}
		\]
		\item Определитель порядка $2 \times 2$:
		\[
		\begin{vmatrix}
		a & b \\
		c & d
		\end{vmatrix}
		= ad - bc
		\]
		Определитель порядка $3 \times 3$:
		\[
		\begin{vmatrix}
		a_{11} & a_{12} & a_{13} \\
		a_{21} & a_{22} & a_{23} \\
		a_{31} & a_{32} & a_{33}
		\end{vmatrix}
		= 
		a_{11}a_{22}a_{33} + a_{12}a_{23}a_{31} + a_{13}a_{21}a_{32} - a_{13}a_{22}a_{31} - a_{12}a_{21}a_{33} - a_{11}a_{23}a_{32}
		\]
		
		\item При транспонировании определитель не меняет знак:
		\[
		\det A = \det A^T
		\]
		Следующие 4 определния равносильны для столбцов
		\item При умножении строки (столбца) на скаляр он выносится за знак определителя:
		\[
		\det\left(
		\begin{matrix}
		A^{(1)}\\
		\vdots\\
		A^{(i)} \cdot a\\
		\vdots\\
		A^{(n)}
		
		\end{matrix}
		\right)
		=
		a \cdot \det\left(
		\begin{matrix}
		A^{(1)}\\
		\vdots\\
		A^{(i)}\\
		\vdots\\
		A^{(n)}
		
		\end{matrix}
		\right)
		\]
		\item При перестановке двух строк (столбцов) определитель меняет знак
		\[
		\det\left(
		\begin{matrix}
		A^{(1)}\\
		\vdots\\
		A^{(i)}\\
		\vdots\\
		A^{(j)}\\
		\vdots\\
		A^{(n)}
		
		\end{matrix}
		\right)
		=
		- \det\left(
		\begin{matrix}
		A^{(1)}\\
		\vdots\\
		A^{(j)}\\
		\vdots\\
		A^{(i)}\\
		\vdots\\
		A^{(n)}
		
		\end{matrix}
		\right)
		\]
		\item Если какую-нибудь строчку (столбец) можно разложить в сумму двух, то определитель раскладывается в сумму определителей
		\[
		\det\left(
		\begin{matrix}
		A^{(1)}\\
		\vdots\\
		A^{(i)} + A'^{(i)}\\
		\vdots\\
		A^{(n)}
		
		\end{matrix}
		\right)
		=
		\det\left(
		\begin{matrix}
		A^{(1)}\\
		\vdots\\
		A^{(j)}\\
		\vdots\\
		A^{(n)}
		
		\end{matrix}
		\right)
		+
		\det\left(
		\begin{matrix}
		A^{(1)}\\
		\vdots\\
		A'^{(i)}\\
		\vdots\\
		A^{(n)}
		
		\end{matrix}
		\right)
		\]
	
		\item При прибавлении к строке (столбцу) другой, умноженной на скаляр, определитель не меняется
		\[
		\det\left(
		\begin{matrix}
		A^{(1)} \\
		\vdots \\
		A^{(i)} + \alpha \cdot A^{(j)}\\
		\vdots \\
		A^{(j)} \\
		\vdots \\
		A^{(n)}
		
		\end{matrix}
		\right)
		=
		\det\left(
		\begin{matrix}
		A^{(1)}\\
		\vdots\\
		A^{(i)}\\
		\vdots\\
		A^{(j)}\\
		\vdots\\
		A^{(n)}
		
		\end{matrix}
		\right)
		\]
		\item 	
		Матрица $A \in  M_n$ – $\textbf{верхнетреугольная}$, если
		$a_{ij} = 0$  при  $i > j$  (т.е. ниже диагонали).
		\[
		\begin{pmatrix}
			a_{11} & \cdots & a_{1n} \\
			0      & \ddots & \vdots \\
			0      & 0      & a_{nn}
		\end{pmatrix}
		\]
		Матрица $A \in  M_n$ – $\textbf{нижнетреугольная}$, если
		$a_{ij} = 0$  при  $i < j$  (т.е. выше диагонали).
		\[
		\begin{pmatrix}
		a_{11} & 0 & 0 \\
		\vdots      & \ddots & 0 \\
		a_{n1}     & \cdots      & a_{nn}
		\end{pmatrix}
		\]
		
		\item Определитель верхне- (нижне-)треугольной матрицы есть произведение диагональных элементов:
		\[
		\det A = \prod_{i = 1}^{n} a_{ii}
		\]
		\item Определитель диагональной матрицы есть произведение элементов, стоящих на диагонали, так как эта матрица одновременно верхне- и нижнетреугольная.
		\[
		\det A = \prod_{i = 1}^{n} a_{ii}
		\]
		Определитель $E \in M_n$ равен 1.
		
		\item Пусть $A, B \in M_n$ имеют вид
		\[
		A =
			\begin{pmatrix}
			D & P\\
			0 & R
			\end{pmatrix},
			\quad
		B =
			\begin{pmatrix}
			D & 0\\
			P & R
			\end{pmatrix}
		\]
		Тогда они будут называться \textit{матрицами с углом нулей}.
		Если матрицы $D$ и $R$ - квадратные, то
		\[
		\det A = \det D \cdot \det R
		\]
		\item Определитель произведения матриц есть произведение определителей матриц
		\[
		\det{AB} = \det{A}\cdot\det{B}
		\]
		
		\[
		\begin{matrix}
		6 & 6 & 6 \\
		6 & 6 & 6 \\
		6 & 6 & 6
		\end{matrix}
		\]
		
		
		
		\item Дополнительным минором к элементу $a_{ij}$ матрицы $A \in \mathrm{M}_n$ называют определитель матрицы, полученной из $A$ удалением $i$-ой строки и $j$-го столбца:
		\[\overline{M}_{ij} = \begin{vmatrix}
		a_{11} & a_{12} & \ldots & a_{1(j-1)} & a_{1(j+1)} & \ldots & a_{1n} \\
		a_{21} & a_{22} & \ldots & a_{2(j-1)} & a_{2(j+1)} & \ldots & a_{2n} \\
		\vdots & \vdots & \ddots & \vdots     & \vdots     & \ddots & \vdots \\
		a_{(i-1)1} & a_{(i-1)2} & \ldots & a_{(i-1)(j-1)} & a_{(i-1)(j+1)} & \ldots & a_{(i-1)n} \\
		a_{(i+1)1} & a_{(i+1)2} & \ldots & a_{(i+1)(j-1)} & a_{(i+1)(j+1)} & \ldots & a_{(i+1)n} \\
		\vdots & \vdots & \ddots & \vdots     & \vdots     & \ddots & \vdots \\
		a_{n1} & a_{n2} & \ldots & a_{n(j-1)} & a_{n(j+1)} & \ldots & a_{nn} \\
		\end{vmatrix}\] 
		
		\item Алгебраическим дополнением элемента $a_{ij}$ матрицы $A \in \mathrm{M}_n$ называют число \[A_{ij} = (-1)^{i+j}\overline{M}_{ij}\]
		
		\item Теорема Лапласа о разложении определителя по строке (столбцу).
		
		Пусть выбрана $i$-я строка матрицы $A \in \mathrm{M}_n$. Тогда определитель матрицы $A$ равен сумме всех элементов строки, умноженных на их алгебраические дополнения:
		\[\det{A} = \sum_{k = 1}^{n} a_{ik}A_{ik}\]
		
		\item Лемма о фальшивом разложении определителя.
		
		Сумма произведений всех элементов некоторой фиксированной строки (столбца) матрицы $A$ на алгебраические дополнения соответствующих элементов любой другой фиксированной строки (столбца) равна нулю.
		
		При фиксированных $i,\ k \in \{1,\ \ldots,\ n\},\ i \ne k$
		$$ \sum_{j=1}^n a_{ij} A_{kj} = 0 $$
		
		При фиксированных $j,\ m \in \{1,\ \ldots,\ n\},\ j \ne m$
		$$ \sum_{i=1}^n a_{ij} A_{im} = 0 $$
		
		\item Матрица $A^{-1} \in M_n$ называется обратной матрицей к $A$, если $AA^{-1} = A^{-1}A = E$.
		
		
		\item Матрица $A \in M_n$ называется невырожденной, если $\det A \ne 0$. 
		
		\item Матрица, присоединённая к $A$ — $\hat{A} = (A_{ij})^T$ — матрица из алгебраических дополнений элементов $A$: 
		\[\hat A = \begin{pmatrix}
		A_{11} & A_{21} & A_{31} & \ldots & A_{n1} \\
		A_{12} & A_{22} & \ldots & \ldots & A_{n2} \\
		A_{13} & \vdots & \ddots &  & \vdots \\
		\vdots & \vdots &  & \ddots & \vdots \\
		A_{1n} & A_{2n} & \ldots & \ldots & A_{nn} \\
		\end{pmatrix}.\]
		
		\item Матрица обратима тогда и только тогда, когда она невырожденна
	
		\item Обратная матрица считается по следующей формуле:
		$$A^{-1} = \displaystyle\frac{1}{\det A} \cdot \hat{A}$$
		\item Произведение матриц обратимо тогда, когда матрицы невырожденны и произведение матриц не вырожденно, тогда
		\[
		(AB)^{-1} = B^{-1}\cdot A^{-1}
		\]
		\item Пусть $A$ — матрица коэффициентов некой СЛУ, в которой количество уравнений равно количеству неизвестных, $\vec b$ — вектор правых частей, матрицы $A_i$ -- матрицы, полученные из $A$ заменой в них $i$-ого столбца на $\vec b$. Если $\det A \not= 0$, то СЛУ имеет единственное решение, которое может быть найдено по формулам 
		$$
		x_i = \frac{\det A_i}{\det A}.
		$$ комплексных
		Эти формулы называются формулами Крамера.
		\item Комплексным числом называется пара чисел $(a, b) \in R^2$, где число $a$ называется действительной частью, $b$ - мнимой.
		
		\subitem 1) Алгебраической формой записи комплексного числа называется запись вида
		\[
		a + bi,
		\]
		где $a$ - действительная, $b$ - мнимая части.
		\subitem 2) Суммой двух комплексных чисел является комплексное число:
		\[
		(a + bi) + (c + di) = (a + c) + (b + d)i
		\]
		\subitem 3) Произведением двух комплексных чисел является комплексное число:
		\[
			(a + bi) \cdot (c + di) = (ac - bd) + (bc + ad)i
		\]
		\subitem 4) Частным двух комплексных чисел является комплексное число:
		\[
		\frac{(a + bi)}{(c + di)} = \frac{ac + bd}{c^2 + d^2} + \frac{bc - ad}{c^2 + d^2}i
		\]
		
		\item Отображение $\mathbb{C} \rightarrow \mathbb{C} : a + bi \mapsto a - bi$ называется (комплексным) сопряжением. Само число $\overline{z} = a - bi$ называется (комплексно) сопряженным к числу $z = a + bi$. 
		
		Свойства комплексного сопряжения:
		\begin{enumerate}
			\item Комплексное сопряжение суммы это сумма комплексных сопряжений:
			\[
			\overline{a + b} = \overline{a} + \overline{b}
			\]
		\end{enumerate}
		
		\item Комплексные числа удобно представлять в виде точек на плоскости:
		\begin{center}
		
		%{\includegraphics{z-as-vector}}
		\begin{tikzpicture}
		\begin{axis}
		[
		axis lines\begin{center}
		 = center,
		xtick={0, 3},
		xticklabels = {$0$, $a$},
		ytick = {2},
		yticklabels = {$b$},
		xlabel=$\mathrm{Re}\,z$,
		ylabel=$\mathrm{Im}\,z$,
		ymin=-1,
		ymax=+3,
		xmin=-1,
		xmax=+4
		]
		\node [right, red] at (axis cs:  3, 2) {$a+bi$};
		\addplot[->] coordinates { (0,0) (3,2) };
		\addplot [dashed, black] coordinates { (3, 0) (3,2) };
		\addplot [dashed, black] coordinates { (0,2) (3,2) };
		\end{axis}
		\end{tikzpicture}
		\end{center}
		В таком представлении сложение комплексных чисел сопоставляется со сложением векторов, а сопряжение — с отражением относительно оси $Ox (\Re z)$.
		
		\item Модулем $|z|$ комплексного числа $z = a + bi$ называется длина соответствующего вектора $|a+bi| = \sqrt{a^2 + b^2}$.
		\subitem 1) $|z| \geqslant 0$, причем $|z| = 0$ тогда и только тогда, когда $z = 0$;
		\subitem 2) $|z + w| \leqslant |z| + |w|$ — неравенство треугольника;
		\subitem 3) $|z|\cdot|\overline{z}| = |z|^2$;
		\subitem 4) $|zw| = |z| \cdot |w|$
		
		
		\item Пусть есть $z \in \C, z \ne 0$. Тогда аргументом числа $z$ будет называться всякий такой угол $\varphi$, что
		\[
		\cos \varphi = \frac{a}{|z|} = \frac{a}{\sqrt{a^2 + b^2}}; \quad \sin \varphi = \frac{b}{|z|} = \frac{b}{\sqrt{a^2 + b^2}}.
		\]
		
		Неформально говоря, аргумент $z$ — это угол между осью $Ox$ и соответствующим вектором.
		\item Запись $z = |z|(\cos\varphi + i\sin\varphi)$ называется тригонометрической формой комплексного числа $z$.
		
		Пусть $z_1 = |z_1|\left(\cos{\varphi_1}+i\sin{\varphi_1}\right)$, $z_2 = |z_2|\left(\cos{\varphi_2} + i\sin{\varphi_2}\right)$. Тогда 
		\[
		z_1z_2 = |z_1||z_2|\left(\cos\left(\varphi_1 + \varphi_2\right) + i\sin\left(\varphi_1 + \varphi_2\right)\right)
		\]
		\[\cfrac{z_1}{z_2} = \cfrac{|z_1|}{|z_2|}\left(\cos\left(\varphi_1-\varphi_2\right) + i\sin\left(\varphi_1 - \varphi_2\right)\right)\]
		
		\item Формула Муавра. Пусть $z = |z|\left(\cos\varphi + i \sin \varphi\right)$. Тогда:
		\[z^n = |z|^n\left(\cos\left(n\varphi\right)+i\sin\left(n\varphi\right)\right) \quad \forall n \in \mathbb{Z}.
		\]
		
		\item 	Корнем $n$-й степени из числа $z$ называется всякое $w\in\mathbb C$: $w^n=z$. То есть
		\[
		\sqrt[n]{z} = \{w\in\mathbb C\ |\ w^n = z\}.
		\]
		
		\[ \sqrt[n]{z} = \Biggl\{\sqrt[n]{|z|}\left(\cos\cfrac{\varphi+2\pi k}{n}+i\sin\cfrac{\varphi+2\pi k }{n}\right)\ \biggl|\ k=0,\ldots,n-1\Biggr\}
		\]
		
		\item Основная теорема алгебры. Всякий многочлен $P\left(z\right) = a_nz^n + a_{n-1}z^{n-1} + \ldots \hm{+} a_1z + a_0$ степени $n$, где $n \geqslant 1$, $a_n \neq 0$, и $a_0,\ldots,a_n \in \mathbb{C}$ имеет корень.
		
		\item Теорема Безу. Пусть есть многочлен $P(x)$ с действительными коэффициентами. Тогда утверждается, что остаток от деления такого многочлена на $(x - c)$ равен $P(c)$
		
		Следствие: всякий многочлен степени $n$ имеет $n$ комплексных корней, с учетом кратности.
		
		\item Кратностью корня многочлена $f(x)$ называется наибольшее $k \in \Z$, такое что
		\[
		f(x) \ \vdots \ (x - c)^k
		\]
		
		\item Векторным пространством $V$ над полем $F$  называется множество векторов $V(F)$, на котором заданы операции сложения и умножения на скаляр:
		\[
		+: V \times V \to V\]\[
		\times: F \times V \to V
		\]
		удовлетворяющими следующим аксиомам:
		\begin{enumerate}
			\item $x + y = y + x; \ \forall x, y \in V$
			\item $(x + y) + z = x + (y + z); \ \forall x, y, z \in V$
			\item $\exists \vec{0}: x + \vec{0} = \vec{0} + x = x$
			\item $\forall x \in V \ \exists (-x): x + (-x) = \vec{0}$
			\item $(\alpha + \beta) \cdot x = \alpha x + \beta x$
			\item $x \cdot (\alpha + \beta) = x\alpha  + x\beta $
			\item $\alpha(\beta x) = (\alpha \beta) x$
			\item $1 \cdot x = x$
		\end{enumerate}
		
		
		\item Подпространством векторного пространства называют подмножество $U \subseteq V$, если:
		\begin{enumerate}
			\item $\vec{0} \in U$
			\item $x, y \in U \Rightarrow x + y \in U$
			\item $\forall \alpha \in F \ x \in U \Rightarrow \alpha x \in U$
		\end{enumerate}
		
		\item Подмножество в $R^n$, являющееся решением системы однородных линейных уравнений вида $A\vec{x} = \vec{0}$ является подпространством. 
		
		\item Всякий вектор вида $\alpha_1v_1 + \ldots + \alpha_nv_n$, где $v \in V, \alpha \in F$ называется линейной комбинацией.
		
		\item Множество всех линейных комбинаций векторов из $S \subseteq V$ называется линейной оболочкой $S$ и обозначается $\langle S \rangle$
		
		\item Линейная оболочка векторного пространства сама является векторным пространством.
		
		\item Векторное пространство называется конечномерным, если оно порождается конечным числом векторов, и бесконечномерным в противном случае.
		
		\item Конечный набор векторов $v_1, \ldots, v_n$ линейно зависим, если 
		\[
		\exists (\alpha_1, \ldots, \alpha_n) \ne (0, \ldots, 0) \Rightarrow \alpha_1v_1 + \ldots + \alpha_nv_n = 0
		\]
		
		\item Конечный набор векторов $v_1, \ldots, v_n$ линейно независим, если он не линейно зависим, то есть 
		\[
		\al_1v_1 + \ldots + \al_nv_n = \vec{0} \ \Rightarrow \ \al_1 = \ldots = \al_n = 0
		\]
		
		\item Возьмём $a_1,\ a_2,\ \ldots,\ a_r$ и $b_1,\ b_2,\ \ldots,\ b_s$ --- две системы векторов, причём $r < s$. Пусть вторая система линейно выражается через первую, то есть $b_i \in \langle a_1,\ a_2,\ \ldots,\ a_r \rangle \forall\ i \in \{1,\ \ldots ,\ s\}$. Тогда система $b_1,\ b_2,\ \ldots,\ b_s$ линейно зависисима.
		
		\item Базисом конечномерного векторного пространства $S$ называется такой набор линейно независимых векторов $v = \{e_1, \ldots e_n\}$, такой что $\langle v\rangle = S$. 
		
		\item Пусть есть векторное пространство $S$. Размерностью этого пространства будет число векторов в базисе этого пространства.
		
		\item Пусть $v_1, \ldots, v_m \in V$ - линейно независимые векторы. Тогда $\forall v \in V$:
		\begin{enumerate}
			\item $	v, v_1, \ldots, v_m$ - также линейно независимые векторы, или
			\item $ v \in \langle v_1, \ldots, v_m\rangle$
		\end{enumerate}
		
		\item Пусть есть векторное пространство $S$ и его подпространство $S' \subseteq S$. Тогда $dim \ S' \leq dim \ S$
		
		\item Фундаментальной системой решений ОСЛУ называется базис решений этой системы.
	\end{enumerate}

	\begin{center}
		\line(1,0){450}
	\end{center}
		
\end{document}
